% use paper, or submit
% use 11 pt (preferred), 12 pt, or 10 pt only

%\documentclass[letterpaper, preprint, paper,11pt]{AAS}	% for preprint proceedings
\documentclass[letterpaper, paper,11pt]{AAS}		% for final proceedings (20-page limit)
%\documentclass[letterpaper, paper,12pt]{AAS}		% for final proceedings (20-page limit)
%\documentclass[letterpaper, paper,10pt]{AAS}		% for final proceedings (20-page limit)
%\documentclass[letterpaper, submit]{AAS}			% to submit to JAS

\usepackage{AAS_packages}
%\usepackage{subfigure} % have subcaption in use instead
%\usepackage[notref,notcite]{showkeys}  % use this to temporarily show labels

\PaperNumber{XX-XXX}

\begin{document}

\title{Real Time Shape Reconstruction for Near Earth Asteroid Landing}

\author{Shankar Kulumani and Taeyoung Lee\thanks{Mechanical and Aerospace Engineering, George Washington University, 800 22nd St NW, Washington, DC 20052, Tel: 202-994-8710, Email: \href{mailto:skulumani@gwu.edu}{\{skulumani,tylee\}@gwu.edu}.}
}


\maketitle{} 		

\begin{abstract}
    Abstract for paper
\end{abstract}

\section{Introduction}\label{sec:introduction}
% Motivation for missions/studying asteroids
Small solar system bodies, such as asteroids and comets, continue to remain a focus of scientific study.
The small size of these bodies prevents the formation of large internal pressures and temperatures which help to preserve the early chemistry of the solar system.
This insight offers additional detail into the formation of the Earth and also of the probable formation of other extrasolar planetary bodies.
Of particular interest are those near-Earth asteroids (NEA) which inhabit heliocentric orbits in the vicinity of the Earth. 
These easily accessible bodies provide attractive targets to support space industrialization, mining operations, and scientific missions.
In spite of the significant interest, and the extensive research by the community, the operation of spacecraft near small bodies remains a challenging problem.

% dynamics are difficult around asteroids
The dynamic environment around asteroids is strongly perturbed and challenging for analysis and mission operations~\cite{scheeres2012}.
Due to their low mass, which in turn causes a low gravitational attraction, asteroids may have irregular shapes.
Furthermore, asteroids may also have a chaotic spin state due to the absorption and emmitance of solar radiation~\cite{rubincam2000}.
As a result, approaches utilizing an inverse square gravitational model do not capture the  true dynamic environment.
In addition, the vast majority of asteroids are difficult to track and characterize using ground based sensors.
Due to their small size, frequently less than \SI{1}{\kilo\meter}, and low albedo, the reflected energy of these asteroids is insufficient for reliable detection or tracking.
Therefore, the dynamics model of the asteroid is relatively coarse prior to insitu measurements from a dedicated spacecraft.
As a result, any spacecraft mission to an asteroid must include the ability to update the dynamic model given insitu measurements and robustness to unmodelled forces.

Another key dynamic consideration is the coupling between rotational and translational states around the asteroid.
The coupling is induced due to the different gravitational forces experienced on various portions of the spacecraft. 
The effect of the gravitational coupling is related to the ratio of the spacecraft size and orbital radius~\cite{hughes2004}.
For operations around asteroids, the ratio is relatively large which causes a much larger coupling between the translational and rotational states.
References~\cite{elmasri2005} and~\cite{sanyal2004a} investigated the copuling of an elastic dumbbell spacecraft in orbit about a central body, but only considered teh case of a spherically symmetric central body.
Furthermore, the spacecraft model is assumed to remain in a planar orbit.
As a result, these developments are not directly applicable to motion about an asteroid, which experiences highly non-Keplerian motion.
Reference~\cite{misra2015b} investigated the effect of coupled motion on long term trajectories around asteroids.
However, the analysis only considered a second order spherical harmonic gravitational potential model. 
Therefore, these results are only valid when far from the asteroid surface and will diverge when used within the Brillouin sphere.

% Gravity model is important and dependent on shape
An accurate gravitational potential model is critical for performing low alititude and/or surface operations around asteroids.
Due to the irregular shape, trajectories will pass within the Brillouin sphere where the typical spherical harmonic model diverges from the true gravitational potential.
With respect to asteroid missions, the standard approach to compute the gravitational potential is the polyhedron potential model~\cite{werner1996}.
The polyhedral potential model provides the exact gravitational potential, and subsequently the gravitational acceleration, for a given triangular faceted shape model of an asteroid.
The method provides the exact potential for a given faceted shape model.
As a result, the accuracy of the gravitational potential is primarily dependent on the accuracy with which the shape modelrepresents the true surface topology.
A high fidelity shape, which necessarily has many vertices and faces, is required for an accurate computation of the gravitational acceleration.
The determination of the shape of the asteroid surface is crucial for both determining the gravitational potential as well as operations near the surface.

% Challenges invovled in operation near asteroids (gravity, shpae, distance)
% generating the shape from the ground is difficult
Prior to the arrival of a spacecraft at an asteroid, Earth based sensors are used to characterize the body.
Using both optical and radar sensors allows for the precise orbit of the asteorid to be determined.
Another vital task is the determination of the asteroid shape from radar data~\cite{hudson1994,busch2011}.
This is a challenging problem as it requires the simultaneous estimation of the asteroid spin state and shape.
Furthermore, determining the shape from radar is currently the only Earth-based technique that can produce detailed three-dimensional shape information of near-Earth objects~\cite{greenberg2015}.
The current approach is based on an estimation scheme which iteratively perturbs a shape to match given radar data.
This computationally intensive approach is only able to capture the gross size and shape and is unable to capture the minute surface features of the asteroid.
Frequently, only a rough triaxial ellipsoid model is possible from the ground and an accurate shape must be determined only after a spacecraft has rendezvouzed with the asteroid.
As a result, on arrival the gravitional environment near the asteroid is unknown as the shape of the asteroid is poorly modelled.

% on arrival spacecraft spend long periods mapping, and depending on misison this might be unallowable
On approach to an asteroid, spacecraft navigation and guidance is primarily based on ground measurements.
After arrival, a spacecraft will generally spend months or years in a mapping mission phase~\cite{kubota2003,cole1998}.
During this period, spacecraft sensors, such as onboard optical telescopes or Light radio Detection and Ranging (LIDAR), are used to characterize the asteroid.
The resulting imagery and range data is transmitted to the ground and the resulting asteroid shape and motion is estimated. 
During this mapping phase the spacecraft must remain in a quiescent state devotes entirely to mapping the surface.
Depending on the mission type this long period of mapping is crucial to determining an ideal landing location, such as the Asteroid Redirect Mission~\cite{gates2015}. 
However, other missions, such as asteroid mitigation, may be severely limited by the time and ground resources required to generate a surface shape.
Furthermore, the long distances involved necessitate on-board autonomy to enable to spacecraft to operate without ground communications.
The dependency on expensive ground based shape reconstruction techniques limit the ability for spacecraft autonomously operate and land at asteroids.

% we want to generate the shape in real time and then use this shape for updated and better control
In this paper, we develop a method to compute the surface shape of an asteroid from range measurements.
Our approach is able to operate in real time and incrementally update the shape model of an asteroid as new range measurements are collected.
This approach allows for the shape to be continually updated as range measurements are used to locally modify the shape estimate instead of a global operation.
Furthermore, this updated shape is then used in a nonlinear controller to enable the tracking of a landing trajectory to the surface.
In contrast to previous work, we explicitly consider the gravitational coupling between orbit and attitude dynamics.
Furthermore, instead of computationally expensive surface reconstruction methods we present a straightforward and conceptually simple method to enable real time shape updates. 


% our approach for paper. Real time method to update asteroid shape model and use updated model in closed loop control

% Benefits and contribution of our approach
In short, this paper presents a method to incrementally update the shape  model of an asteroid from range measurements. 
Our approach alleviates the need for a dedicated mapping phase as the spacecraft is able to update its shape model in real time and without expensive computations.
This type of approach allows for the spacecraft to maneuver and land on the asteroid immediately upon arrival rather than spending several months mapping the surface.
This updated shape model is then used to in a nonlinear controller to track a desired state trajectory for the dynamics of a rigid body spacecraft.
The dynamics are developed on the nonlinear manifold of rigid body motions, namely the special euclidean group.
This formulation is based on an intrinsic geometric description of the motion and accurately captures the coupling between orbit and attitude dynamics. 
The presented approach allows for a spacecraft to transition directly from arrival to the surface while reconstructing the surface shape in real time.


\section{Problem Formulation}\label{sec:problem}

In this paper, we consider the motion of a dumbbell model of spacecraft around an asteroid.
The dumbbell model captures the important interactions of the coupling between orbital and attitude dynamics.
In this model, the spacecraft consists of two masses connected by a massless rod.
The asteroid is modeled as a constant density polyhedron with constant, and known, spin about its maximum moment of inertia. 
Without loss of generality, we define body fixed frames for both the spacecraft and asteroid which are aligned with the principle axes of each body and originate at their respective center of mass. 
The kinematics of the dumbbell and asteroid are described in the inertial frame by
\begin{itemize}
    \item \( \vecbf{x} \) - the position of the center of mass of the dumbbell spacecraft represented in the inertial frame \( \vecbf{e}_i\)
    \item \( R \) - the rotation matrix which transforms vectors defined in the spacecraft fixed frame, \( \vecbf{b}_i \), to the inertial frame, \( \vecbf{e}_i \)
    \item \( \vecbf{\Omega} \) - the angular velocity of the spacecraft body fixed frame relative to the inertial frame and represented in the dumbbell body fixed frame \( \vecbf{b}_i \)
    \item \( R_A \) - the rotation matrix which transforms vectors defined in the asteroid fixed frame, \( \vecbf{f}_i \), to the inertial frame, \( \vecbf{e}_i \)
\end{itemize}
In this work, we assume that the asteroid is much more massive than the spacecraft and it's motion is not affected by that of the spacecraft.
This assumption allows us to treat the motion of the vehicle independently from that of the asteroid. 

Using Hamiltons principle gives the inertial equations of motion of the dumbbell spacecraft as
\begin{align}
    \dot{\vb{x}} &= \vb{v}, \\
    \parenth{m_1 + m_2} \dot{\vecbf{v}} &= m_1 R_A \deriv{U}{\vecbf{z}_1} + m_2 R_A \deriv{U}{\vecbf{z}_2}, \\
    \dot{R} &= R S(\vb{\Omega}) , \\
    J \dot{\vecbf{\Omega}} + \vecbf{\Omega} \times J \vecbf{\Omega} &= \vecbf{M}_1 + \vecbf{M}_2.
\end{align}
The vectors \( \vecbf{z}_1 \) and \( \vecbf{z}_2\) define the position of the dumbbell masses as represented in the asteroid fixed frame and are defined as
\begin{align}
    \vecbf{z}_1 &= R_A^T \parenth{\vecbf{x} + R \vecbf{\rho}_1} , \\
    \vecbf{z}_2 &= R_A^T \parenth{\vecbf{x} + R \vecbf{\rho}_2}.
\end{align}
The gravitational moment on the dumbbell \( \vecbf{M}_i\) is defined as
\begin{align}
    \vecbf{M}_i = m_i \parenth{S(R_A^T \vb{\rho}_i) R^T \deriv{U}{\vb{z}_i}}.
\end{align}
where the polyhedron potential is defined as 
\begin{align}
    U(\vecbf{r}) &= \frac{1}{2} G \sigma \sum_{e \in \text{edges}} \vecbf{r}_e \cdot \vecbf{E}_e \cdot \vecbf{r}_e \cdot L_e - \frac{1}{2}G \sigma \sum_{f \in \text{faces}} \vecbf{r}_f \cdot \vecbf{F}_f \cdot \vecbf{r}_f \cdot \omega_f,
\end{align}
and \( \vecbf{r}_e\) and \(\vecbf{r}_f \) are the vectors from the spacecraft to any point on the respective edge or face, \( G\) is the universal gravitational constant, and \( \sigma \) is the constant density of the asteroid.
The position of each mass \(m_i\) of the dumbbell is defined in the dumbbell fixed frame by the vector \(\vb{\rho}_i\). 

\subsection{Shape Reconstruction}

We assume that the spacecraft contains an initial estimate for the shape of the asteroid.
This shape can be a coarse estimate computed from radar measurements or it can be a triaxial ellipsoid based on the semimajor axes of the asteroid, such as that shown in~\cref{fig:initial_ellipsoid} which represents the maximum axes for asteroid Castalia.
This model is defined as a sequence of vertices, \( \vb{v}_i\), in the asteroid fixed frame and triangular facets, \( f_i = \bracket{\vb{v}_1, \vb{v}_2, \vb{v}_3}\), which are defined by the three vertices making up the facet.
This shape model is a widely used format for defining three-dimensional shapes in computer applications and has become the standard method for representing asteroid shapes~\cite{neese2004}.
\begin{figure}
    \centering
    \includegraphics[width=0.5\textwidth]{example-image-golden}
    \caption{Triaxial Ellipsoid Triangular Shape Model~\label{fig:initial_ellipsoid}}
\end{figure}

We assume the spacecraft contains a range sensor, such as LIDAR, that allows for the accurate measurement of the relative distance between the spacecraft and asteroid~\cite{zuber1997,zuber2000}.
This type of sensor measures the round-trip time for a pulse of energy to leave the spacecraft, reflect off the surface, and return to a collector on board.
Given the time total time of flight the distance can be accurately computed using \( d = \frac{t_f}{2 c} \) where \( c \) is the constant speed of light.
Assuming accurate knowledge of the pointing direction of the spacecraft we can compute a direction from the spacecraft to the measurement location on the surface.
The output of this sensor is a vector, \( \vb{d}_i \), defined in the spacecraft fixed frame which gives the direction to a measurement point on the surface of the asteroid. 
Using the state of the asteroid, we can transform this measurement to the asteroid fixed frame using the simple t transformation
\begin{align*}
    \vb{p}_i = R_A^T R \vb{d}_i .
\end{align*}
\begin{figure}
    \centering
    \includegraphics[width=0.5\textwidth]{example-image-golden}
    \caption{Simulated LIDAR measurements of asteroid Castalia~\label{fig:lidar_example}}
\end{figure}
Given many measurements, \( \vb{p}_i \), of the asteroid surface we can efficiently update our initial shape estimate to that of the true surface.

Our algorithm applies a Bayesian framework to radially modify each vertex \( \vb{v}_i\) of the shape estimate based on measurement \( \vb{p}_i\). 
This approach alleviates much of complexity of incorporating new vertices or surface triangulation common in surface reconstruction methods~\cite{berg2008}.
This assumption means that the total number of vertices of the shape model is fixed.
However, additional detail, in the form of additional vertices, is possible by using standard mesh subdivision algorithms~\cite{orourke1998}.

The radial distance of each vertex is assumed to be distributed according to the Gaussian distribution
\begin{align*}
    \norm{\vb{v}_i} \sim \mathcal{N}(r_0, w_0^2)
\end{align*}
where \( r_0 \) is the initial estimate of the radial distance of vertex \( \vb{v}_i\) and \( w_0 \) is the initial variance, or confidence, in the radial distance.
The radial distance of each measurement \( \vb{p}_i\) is also assumed to be distributed according to the Gaussian distribution
\begin{align*}
    \norm{\vb{p}_i} \sim \mathcal{N}(r_{m,i}, w_{m,i}^2)
\end{align*}
where \( r_m = \norm{\vb{p}_i} \) defines the true radial distance of the surface vector measurement and \( w_m\) defines the variance of the measurement with respect to vertex \( \vb{v}_i\).

The variance for each measurement vector is assumed to be related to the ``distance'' from the measurement to vertex \( \vb{v}_i \).
Here we use the geodesic distance to parameterize the difference, and hence  uncertainty, or associating the measurement with a given vertex.
From spherical trigonometry the central angle between measurement \( \vb{p}_i \) and vertex \( \vb{v}_i \) of the shape estimate is
\begin{align}
    \Delta \sigma_i = \arctan \parenth{\frac{\norm{\vb{p}_i \times \vb{v}_i}}{\vb{p}_i \cdot \vb{v}_i }}.
\end{align}
The variance of measurement \( \vb{p}_i \) with respect to vertex \( \vb{v}_i \) is then defined as the arc length as
\begin{align}
    w_{m, i} = \norm{\vb{p}_i} \Delta \sigma .
\end{align}
This approach relates the uncertainty of the measurement with the distance to a given vertex.
As a result, measurements which are far from a vertex, where \( \Delta \sigma \) is large, will tend to have a larger variance and hence uncertainty. 
From Bayes' theorem, the a posterior probability is
\begin{align}
    p(\norm{\vb{v}_i} | \norm{\vb{p}_i}) = \frac{p(\norm{\vb{p}_i} | \norm{\vb{v}_i}) p(\norm{\vb{v}_i})}{p(\norm{\vb{p}_i}} \propto p(\norm{\vb{p}_i} | \norm{\vb{v}_i}) p(\norm{\vb{v}_i}).
\end{align}
From the properties of a Gaussian the a posterior probability given a measurement is also distributed according to a Gaussian distribution
\begin{align}
    \mathcal{N} \parenth{\frac{w_{m, i}^2 r_0 + w_0^2 r_{m, i}}{w_0^2 + w_{m, i}^2} , \frac{w_0^2 + w_{m, i}^2}{w_0^2 + w_{m, i}^2}} .
\end{align}
The a posterior probability conditioned on the measurement is the weighted average of the prior knowledge and the measurement. 
Measurements that are far from the vertex will have a high uncertainty or variance and will have a reduced impact on the vertex.
Additional measurements are incoporated using a weighted average of prior belief and the measurement uncertainty.

In order to improve the computational efficiency measurement updates are assumed to be local in nature.
Instead of applying a measurement to all vertices of the mesh the measurement is only applied to the vertices which are within a specified region of the measurement. 

When a measurement is far away from a given vertex, the influence of the measurement on the radial position of the vertex is diminished.

Compute angle between measurement and all vertices of shape

Figure out geodesic distance

Update both radius of vertex and weight

\section{Preliminary Results}

Describe the simulation

Assume initial ellipsoid

Vertices are taken from the shape model and serve to emulate measurements from a lidar

Each vertex is added in real time and only those within a certain area/angle are considered

Reconstruction of asteroid Castalia

\section{Expected Results and Significance}
Fast way to update mesh

Final paper will demonstrate how to use this updated mesh with the gravity model in the controller

Also compare the gravity of  estimate and truth model
\bibliographystyle{AAS_publication} 
\bibliography{library}

\end{document}
